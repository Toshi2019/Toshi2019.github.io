%================================================
%    この tex ファイルは2021年度立命館大学数学研究会機関紙
%    『方程』の記事作成テンプレートです. 
%================================================

% -----------------------
% preamble
% -----------------------
% ここから本文 (\begin{document}) までの
% ソースコードに変更を加えた場合は
% 編集者まで連絡してください. 
% Don't change preamble code yourself. 
% If you add something
% (usepackage, newtheorem, newcommand, renewcommand),
% please tell it 
% to the editor of institutional paper of RUMS.

% ------------------------
% documentclass
% ------------------------
\documentclass[11pt, a4paper, dvipdfmx]{jsarticle}

% ------------------------
% usepackage
% ------------------------
\usepackage{algorithm}
\usepackage{algorithmic}
\usepackage{amscd}
\usepackage{amsfonts}
\usepackage{amsmath}
\usepackage[psamsfonts]{amssymb}
\usepackage{amsthm}
\usepackage{ascmac}
\usepackage{color}
\usepackage{enumerate}
\usepackage{fancybox}
\usepackage[stable]{footmisc}
\usepackage{graphicx}
\usepackage{listings}
\usepackage{mathrsfs}
\usepackage{mathtools}
\usepackage{otf}
\usepackage{pifont}
\usepackage{proof}
\usepackage{subfigure}
\usepackage{tikz}
\usepackage{verbatim}
\usepackage[all]{xy}

\usetikzlibrary{cd}



% ================================
% パッケージを追加する場合のスペース 

%=================================


% --------------------------
% theoremstyle
% --------------------------
\theoremstyle{definition}

% --------------------------
% newtheoem
% --------------------------

% 日本語で定理, 命題, 証明などを番号付きで用いるためのコマンドです. 
% If you want to use theorem environment in Japanece, 
% you can use these code. 
% Attention!
% All theorem enivironment numbers depend on 
% only section numbers.
\newtheorem{Axiom}{公理}[section]
\newtheorem{Definition}[Axiom]{定義}
\newtheorem{Theorem}[Axiom]{定理}
\newtheorem{Proposition}[Axiom]{命題}
\newtheorem{Lemma}[Axiom]{補題}
\newtheorem{Corollary}[Axiom]{系}
\newtheorem{Example}[Axiom]{例}
\newtheorem{Claim}[Axiom]{主張}
\newtheorem{Property}[Axiom]{性質}
\newtheorem{Attention}[Axiom]{注意}
\newtheorem{Question}[Axiom]{問}
\newtheorem{Problem}[Axiom]{問題}
\newtheorem{Consideration}[Axiom]{考察}
\newtheorem{Alert}[Axiom]{警告}
\newtheorem{Fact}[Axiom]{事実}


% 日本語で定理, 命題, 証明などを番号なしで用いるためのコマンドです. 
% If you want to use theorem environment with no number in Japanese, You can use these code.
\newtheorem*{Axiom*}{公理}
\newtheorem*{Definition*}{定義}
\newtheorem*{Theorem*}{定理}
\newtheorem*{Proposition*}{命題}
\newtheorem*{Lemma*}{補題}
\newtheorem*{Example*}{例}
\newtheorem*{Corollary*}{系}
\newtheorem*{Claim*}{主張}
\newtheorem*{Property*}{性質}
\newtheorem*{Attention*}{注意}
\newtheorem*{Question*}{問}
\newtheorem*{Problem*}{問題}
\newtheorem*{Consideration*}{考察}
\newtheorem*{Alert*}{警告}
\newtheorem{Fact*}{事実}


% 英語で定理, 命題, 証明などを番号付きで用いるためのコマンドです. 
% If you want to use theorem environment in English, You can use these code.
%all theorem enivironment number depend on only section number.
\newtheorem{Axiom+}{Axiom}[section]
\newtheorem{Definition+}[Axiom+]{Definition}
\newtheorem{Theorem+}[Axiom+]{Theorem}
\newtheorem{Proposition+}[Axiom+]{Proposition}
\newtheorem{Lemma+}[Axiom+]{Lemma}
\newtheorem{Example+}[Axiom+]{Example}
\newtheorem{Corollary+}[Axiom+]{Corollary}
\newtheorem{Claim+}[Axiom+]{Claim}
\newtheorem{Property+}[Axiom+]{Property}
\newtheorem{Attention+}[Axiom+]{Attention}
\newtheorem{Question+}[Axiom+]{Question}
\newtheorem{Problem+}[Axiom+]{Problem}
\newtheorem{Consideration+}[Axiom+]{Consideration}
\newtheorem{Alert+}{Alert}
\newtheorem{Fact+}[Axiom+]{Fact}
\newtheorem{Remark+}[Axiom+]{Remark}

% ----------------------------
% commmand
% ----------------------------
% 執筆に便利なコマンド集です. 
% コマンドを追加する場合は下のスペースへ. 

% 集合の記号 (黒板文字)
\newcommand{\NN}{\mathbb{N}}
\newcommand{\ZZ}{\mathbb{Z}}
\newcommand{\QQ}{\mathbb{Q}}
\newcommand{\RR}{\mathbb{R}}
\newcommand{\CC}{\mathbb{C}}
\newcommand{\PP}{\mathbb{P}}
\newcommand{\KK}{\mathbb{K}}


% 集合の記号 (太文字)
\newcommand{\nn}{\mathbf{N}}
\newcommand{\zz}{\mathbf{Z}}
\newcommand{\qq}{\mathbf{Q}}
\newcommand{\rr}{\mathbf{R}}
\newcommand{\cc}{\mathbf{C}}
\newcommand{\pp}{\mathbf{P}}
\newcommand{\kk}{\mathbf{K}}

% 特殊な写像の記号
\newcommand{\ev}{\mathop{\mathrm{ev}}\nolimits} % 値写像
\newcommand{\pr}{\mathop{\mathrm{pr}}\nolimits} % 射影

% スクリプト体にするコマンド
%   例えば {\mcal C} のように用いる
\newcommand{\mcal}{\mathcal}

% 花文字にするコマンド 
%   例えば {\h C} のように用いる
\newcommand{\h}{\mathscr}

% ヒルベルト空間などの記号
\newcommand{\F}{\mcal{F}}
\newcommand{\X}{\mcal{X}}
\newcommand{\Y}{\mcal{Y}}
\newcommand{\Hil}{\mcal{H}}
\newcommand{\RKHS}{\Hil_{k}}
\newcommand{\Loss}{\mcal{L}_{D}}
\newcommand{\MLsp}{(\X, \Y, D, \Hil, \Loss)}

% 偏微分作用素の記号
\newcommand{\p}{\partial}

% 角カッコの記号 (内積は下にマクロがあります)
\newcommand{\lan}{\langle}
\newcommand{\ran}{\rangle}



% 圏の記号など
\newcommand{\Set}{{\bf Set}}
\newcommand{\Vect}{{\bf Vect}}
\newcommand{\FDVect}{{\bf FDVect}}
\newcommand{\Ring}{{\bf Ring}}
\newcommand{\Ab}{{\bf Ab}}
\newcommand{\Mod}{\mathop{\mathrm{Mod}}\nolimits}
\newcommand{\CGA}{{\bf CGA}}
\newcommand{\GVect}{{\bf GVect}}
\newcommand{\Lie}{{\bf Lie}}
\newcommand{\dLie}{{\bf Liec}}



% 射の集合など
\newcommand{\Map}{\mathop{\mathrm{Map}}\nolimits}
\newcommand{\Hom}{\mathop{\mathrm{Hom}}\nolimits}
\newcommand{\End}{\mathop{\mathrm{End}}\nolimits}
\newcommand{\Aut}{\mathop{\mathrm{Aut}}\nolimits}
\newcommand{\Mor}{\mathop{\mathrm{Mor}}\nolimits}

% その他便利なコマンド
\newcommand{\dip}{\displaystyle} % 本文中で数式モード
\newcommand{\e}{\varepsilon} % イプシロン
\newcommand{\dl}{\delta} % デルタ
\newcommand{\pphi}{\varphi} % ファイ
\newcommand{\ti}{\tilde} % チルダ
\newcommand{\pal}{\parallel} % 平行
\newcommand{\op}{{\rm op}} % 双対を取る記号
\newcommand{\lcm}{\mathop{\mathrm{lcm}}\nolimits} % 最小公倍数の記号
\newcommand{\Probsp}{(\Omega, \F, \P)} 
\newcommand{\argmax}{\mathop{\rm arg~max}\limits}
\newcommand{\argmin}{\mathop{\rm arg~min}\limits}





% ================================
% コマンドを追加する場合のスペース 

% =================================





% ---------------------------
% new definition macro
% ---------------------------
% 便利なマクロ集です

% 内積のマクロ
%   例えば \inner<\pphi | \psi> のように用いる
\def\inner<#1>{\langle #1 \rangle}

% ================================
% マクロを追加する場合のスペース 

%=================================





% ----------------------------
% documenet 
% ----------------------------
% 以下, 本文の執筆スペースです. 
% Your main code must be written between 
% begin document and end document.
% ---------------------------

\title{方程2021テンプレート}
\author{筆者名}
\date{}
\begin{document}
\maketitle

% abstract:記事の内容を要約する環境です(使用の有無は任意)
\begin{abstract}
    このドキュメントは
    2021年度立命館大学数学研究会機関紙
    『方程』の記事作成テンプレートです. 
    このテンプレートを書き換えて
    方程の記事を作成してください. 
\end{abstract}


\section{"章"のコマンド (番号付き) }

\subsection{"節"のコマンド (番号付き) }

ここに本文を書きます. 


\subsection{数式環境}

数式環境の使い方を復習しておきます. 
本文中で数式混じりの文を書くには \$ 2つで挟んで
\begin{screen}
    \begin{verbatim}
        2次方程式
            $ax^2 + bx + c = 0$
        の解の公式は\end{verbatim}
\end{screen}
のように書きます. すると
\begin{screen}
    \begin{quote}
        2次方程式
            $ ax^2 + bx + c = 0 $
        の解の公式は
    \end{quote}
\end{screen}
のように表示されます. 他にも色々できるので, 
コマンドがわからなくなったら, 例えば
\verb|TeX 分数|
などで検索してみてください. 
\verb|\frac{}{}|
というコマンドが見つかるはずです. 
% ソースコード内で行を空けると段落が変わって表示されます

数式を紙面の中央で表示するには
\verb|align|
や
\verb|align*|
を用います. 
\verb|align|
は式番号がつきます: 
\begin{align}
    x = \frac{-b \pm \sqrt{b^2 - 4ac}}{2a}
\end{align}
一方
\verb|align*|
では
\begin{align*}
    x = \frac{-b \pm \sqrt{b^2 - 4ac}}{2a}
\end{align*}
のように式番号がつきません. \LaTeX では基本的に
\verb|*|つきのコマンドには番号がつかないと
思っておいて良いでしょう. 

\subsection{定理環境}

定理環境の使い方を説明します. 
\begin{screen}
    \begin{verbatim}
        \begin{Proposition}[定理名]
            主張. 
        \end{Proposition}\end{verbatim}
\end{screen}
のようにかくと
\begin{Proposition}[定理名]
    主張. 
\end{Proposition}
のように表示されます. 証明も
\begin{screen}
    \begin{verbatim}
        \begin{proof}
            証明証明証明証明証明証明証明証明証明
        \end{proof}\end{verbatim}
\end{screen}
のようにかくと
\begin{proof}
    証明証明証明証明証明証明証明証明証明
\end{proof}
のように表示されます. \textit{Proof}の部分は
\begin{screen}
    \begin{verbatim}
        \begin{proof}[証明]
            証明証明証明証明証明証明証明証明証明
        \end{proof}\end{verbatim}
\end{screen}
とすれば
\begin{proof}[証明]
    証明証明証明証明証明証明証明証明証明
\end{proof}
のように変更できます. 

以下, サンプルです. 

\begin{Proposition}[はさみうちの原理(squeeze theorem)]
    $a$と$b$を実数とする. 任意の自然数$n\geqq1$に対し
    $\dip|a-b|<\frac{1}{n}$ならば, $a=b$ である. 
\end{Proposition}

\begin{proof}[{\bf 証明}]
    $|a-b|>0$ だったとする. 
    このとき, $\dip\frac{1}{|a-b|}>0$ は実数なので, 
    アルキメデスの公理より
    $\dip \frac{1}{\left|a-b\right|}\leqq n$ 
    をみたす自然数$n>0 \Leftrightarrow n\geqq1$
    が存在する. この不等式の逆数をとると, 
    $\dip \frac{1}{n}\leqq |a-b|$
    となるが, 仮定より$\dip|a-b| < \frac{1}{n}$
    なので
    $\dip \frac{1}{n}\leqq|a-b| < \frac{1}{n}$
    となりムジュン. したがって$|a-b|=0$であり, $a=b$.
  \end{proof}

\section*{章 (番号なし) }

\subsection*{節 (番号なし) }

本文. 

実際の紙面に反映されないソースコード中でも
こまめに改行したり, 
インデントしたりしておくと編集しやすいと思います. 
例えば, 
\begin{screen}
    $(a_n)$ を自然数列で, 任意の $n \geqq 0$ に対し, 
    $a_n = 0$ か $a_n = 1$ のどちらかであるものとする. 
    このとき, 実数$b$で, 全ての自然数$m \geqq 0$に対し
    \begin{eqnarray}
        \sum_{n=1}^{m}\frac{a_n}{2^n} 
        \leqq b 
        \leqq \sum_{n=1}^{m}\frac{a_n}{2^n} + \frac{1}{2^m}
    \end{eqnarray}
    をみたすものが存在する. 
\end{screen}
という文章は, ソースコード中では
\begin{screen}
    \begin{verbatim}
        $(a_n)$ を自然数列で, 任意の $n \geqq 0$ に対し, 
        $a_n = 0$ か $a_n = 1$ のどちらかであるものとする. 
        このとき, 実数$b$で, 全ての自然数$m \geqq 0$に対し
        \begin{eqnarray}
            \sum_{n=1}^{m}\frac{a_n}{2^n} 
            \leqq b 
            \leqq \sum_{n=1}^{m}\frac{a_n}{2^n} + \frac{1}{2^m}
        \end{eqnarray}
        をみたすものが存在する. \end{verbatim}
\end{screen}
のように書いています. 

\subsection*{コマンド集のサンプル}
プリアンブル(ソースコードの上の方に書いてあるコード)
のコマンドを使った場合のサンプルを書いておきます. 
\begin{itemize}
    \item $\NN \QQ \RR \CC \PP \KK$
    \item $\nn \qq \rr \cc \pp \kk$
    \item $\ev_x:\Map(X,Y)\to Y$
    \item $\pr_1:X\times Y \to X$
    \item $\forall \e >0 \quad \exists \dl > 0 \quad 
          \forall x \in U_{\dl}(a) \quad |f(x) - f(a)| < \e$
    \item ${\mcal C} \simeq \Mod(R)$
    \item ${\h C}^\op \to \Set$
    \item $\inner<\pphi | \psi>$
\end{itemize}





%===============================================
% 参考文献スペース
%===============================================
\begin{thebibliography}{20} 
    \bibitem{ひ1} 筆者, 『本の名前』, 出版社, 出版年.
    \bibitem{AB1} A.\ Author, B.\ Buthor, \textit{Title of The Book}, Publisher, 9999.
\end{thebibliography}

%===============================================


\end{document}
