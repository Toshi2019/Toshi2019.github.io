\documentclass[dvipdfmx,12pt,aspectratio=169]{beamer}% dvipdfmxしたい
\usepackage{bxdpx-beamer}% dvipdfmxなので必要
\usepackage{pxjahyper}% 日本語で'しおり'したい
\usepackage{minijs}% min10ヤダ
\renewcommand{\kanjifamilydefault}{\gtdefault}
\renewcommand{\emph}[1]{{upshape\bfseries #1}}

\usepackage{amsmath}
\usepackage{amsthm}
\usepackage{tikz}
\usepackage{color}
\usepackage{ascmac}
\usepackage{amsfonts}
\usepackage{mathrsfs}
\usepackage{mathtools}
\usepackage{amssymb}
\usepackage{graphicx}
\usepackage{fancybox}
\usepackage{enumerate}
\usepackage{verbatim}
\usepackage{subfigure}
\usepackage{proof}
\usepackage{listings}
\usepackage{otf}
\usepackage[all]{xy}
\usepackage{amscd}
%\usepackage[dvipdfmx]{hyperref}

\usepackage{xcolor}
\definecolor{darkgreen}{rgb}{0,0.45,0} 
\definecolor{darkred}{rgb}{0.75,0,0}
\definecolor{darkblue}{rgb}{0,0,0.6} 
\hypersetup{
    colorlinks=true,
    citecolor=darkgreen,
    linkcolor=darkblue,
    urlcolor=darkblue,
}

\usetikzlibrary{positioning}


%\usepackage{pxjahyper}

%\renewcommand{\proofname}{\bf 証明} % 「証明」の見出しを日本語にする
\newcommand{\pr}{\mathop{\mathrm{pr}}\nolimits} % 射影の記号を斜字体にしない
\newcommand{\id}{\mathop{\mathrm{id}}\nolimits} % 恒等写像
\newcommand{\Ob}{\mathop{\mathrm{Ob}}\nolimits}
%\numberwithin{equation}{section} % 式番号を「(3.5)」のように印刷
\newcommand{\Hom}{\mathop{\mathrm{Hom}}\nolimits}
\newcommand{\Mod}{\mathop{\mathrm{Mod}}\nolimits}
\newcommand{\End}{\mathop{\mathrm{End}}\nolimits}
\newcommand{\Aut}{\mathop{\mathrm{Aut}}\nolimits}
\newcommand{\Mor}{\mathop{\mathrm{Mor}}\nolimits}
\newcommand{\Map}{\mathop{\mathrm{Map}}\nolimits}

\newcommand{\dip}{\displaystyle} % 本文中で数式モード



\usetheme{Singapore}
\usecolortheme{rose}
\setbeamertemplate{navigaton symbols}{}
%\usepackage{graphicx,xcolor}

\begin{document}
    
\title{2021年度 方程について}
\author{大柴 寿浩}


\begin{frame}
    \titlepage
\end{frame}

\begin{frame}\frametitle{内容}
    \tableofcontents
\end{frame}

\section{方程について}

\begin{frame}\frametitle{方程について}
    \begin{itemize}
        \item 数学研究会の機関紙, 年一回発行
        \item 一人一記事, 自分の勉強したことをまとめる
        \item テーマは自由
    \end{itemize}
\end{frame}

\begin{frame}\frametitle{ページ数}
    \begin{itemize}
        \item 1回生はページ数自由
        \item 2回生以上は4ページ以上は書いて欲しい
    \end{itemize}
\end{frame}


\section{TeX}

\begin{frame}\frametitle{TeX (LaTeX) とは}
    \begin{itemize}
        \item 数式等に特化した文書作成ツール (言語)
        \item 数学系の論文は TeX ファイルでの提出がキホン
        \item スライドも作れる (コレも TeX)
    \end{itemize}
\end{frame}

\begin{frame}\frametitle{TeX を始める}
    だいたいの場合, 次のどちらか
    \begin{itemize}
        \item 自前の環境を作る
        \item クラウドサービスを使う 
        (\href{https://cloudlatex.io/ja}{Cloud LaTeX}, 
        \href{https://ja.overleaf.com/}{Overleaf} 等)
    \end{itemize}
    入門書\cite{1}はおすすめ. 
    
    \href{https://toshi2019.github.io}{ホームページ}に
    参考になる文書がまとめてあります

    テンプレート参照
\end{frame}

\section{作成の流れ}

\begin{frame}\frametitle{作成の流れ}
    \begin{tikzpicture}
        \draw (-0.5,0) -- (10.5,0);

        \draw (0,0) -- (0,4);
        \draw (2,0) -- (2,4);
        \draw (4,0) -- (4,4);
        \draw (6,0) -- (6,4);
        \draw (8,0) -- (8,4);
        \draw (10,0) -- (10,4);
        \draw (0,0)node[below]{2021年10月};
        \draw (2,0)node[below]{11月};
        \draw (4,0)node[below]{12月};
        \draw (4,-0.5)node[below]{締切};
        \draw (6,0)node[below]{2022年1月};
        \draw (8,0)node[below]{2月};
        \draw (10,0)node[below]{3月};
        \draw (10,-0.5)node[below]{完成};

        \fill[cyan!20, rounded corners] (0,2.7)rectangle(4,3.7);
        \draw[cyan, rounded corners = 5pt] (0,2.7)rectangle(4,3.7);
        \draw (2,3.2) node {{\color{darkblue} 執筆}};

        \fill[cyan!20, rounded corners] (4,2.7)rectangle(7,3.7);
        \draw[cyan, rounded corners = 5pt] (4,2.7)rectangle(7,3.7);
        \draw (5.5,3.2) node {{\color{darkblue} 修正}};
        
        \fill[black!0, rounded corners] (3,1.5)rectangle(7,2.5);
        \draw[rounded corners = 5pt] (3,1.5)rectangle(7,2.5);
        \draw (5,2) node {査読};

        \fill[magenta!40, rounded corners] (0,0.3)rectangle(3,1.3);
        \draw[magenta, rounded corners = 5pt] (0,0.3)rectangle(3,1.3);
        \draw (1.5,0.8) node {{\color{darkred} 割振}};

        \fill[magenta!40, rounded corners] (3,0.3)rectangle(9,1.3);
        \draw[magenta, rounded corners = 5pt] (3,0.3)rectangle(9,1.3);
        \draw (6,0.8) node {{\color{darkred} 編集}};


        \fill[cyan!20, rounded corners] (-0.5,4.5)rectangle(0,5);
        \draw[cyan, rounded corners = 5pt] (-0.5,4.5)rectangle(0,5);
        \draw (0,4.75) node [right]{{\color{darkblue}個人}};

        \fill[black!0, rounded corners] (1.25,4.5)rectangle(1.75,5);
        \draw[black, rounded corners = 5pt] (1.25,4.5)rectangle(1.75,5);
        \draw (1.75,4.75) node [right]{査読者};
   
        \fill[magenta!40, rounded corners] (3.4,4.5)rectangle(3.9,5);
        \draw[magenta, rounded corners = 5pt] (3.4,4.5)rectangle(3.9,5);
        \draw (3.9,4.75) node [right]{{\color{darkred}編集}};   
    \end{tikzpicture}
\end{frame}



\begin{frame}\frametitle{締切}
    締切は今年いっぱい, つまり
    \begin{center}
        \Large{2021年12月31日 23:59 (日本時間)}
    \end{center}
    とします
\end{frame}


\begin{frame}\frametitle{提出方法}
    次のどれかの方法で提出してください
    \begin{itemize}
        \item 提出フォーム: \url{https://forms.gle/AjypdfXJjCSXhvhy8}
        \item メール: \url{houtei2021@gmail.com}
        \item 数理科 slack のアカウントにDM
    \end{itemize}
\end{frame}

\section{編集部}

\begin{frame}\frametitle{方程の編集者を募集します}
    こんな人に来てもらいたい
    \begin{itemize}
        \item PC まわりに詳し (くなりた) い
        \item \LaTeX いじるの苦じゃない
    \end{itemize}
\end{frame}

\begin{frame}\frametitle{方程の編集者を募集します}
    編集部に入ると
    \begin{itemize}
        \item \LaTeX に対する理解が深まる
        \item 表紙のデザインを決められる
        \item (提出が\ldots)
    \end{itemize}
\end{frame}

\section{参考文献}

\begin{frame}[allowframebreaks]{参考文献}
    \begin{thebibliography}{木下99}\beamertemplatetextbibitems
        \bibitem[美文書]{1} 奥村晴彦, 黒木裕介
        『[改訂第8版]LaTeX2$\varepsilon$美文書作成入門』
        技術評論社, 2020.
    \end{thebibliography}
\end{frame}


\end{document}